\chapter{Conclusions}
	Overall the project represented a successful way to address the issue of being an expert to exploit the \textit{DOREMUS} informations. Thanks to our work, a user can simply chat to the bot to know informations about artists, works and events.\\\\
	The project involved us in a challenging series of tasks which weren't limited to the pure implementation of the queries but also to engage the user a conversational experience, try to understand the queries even if they contain misspelled words and automatically detect the language used by the user: this meant to dive into some \textit{Node.js} code, compare different libraries, think about smart solutions. The project brought us also to perform two pull requests to the author of the \texttt{botkit-middleware-dialogflow}: the author was happy to accept our solutions and merged them into the main project; this meant that our work wasn't isolated to the mere execution of the required tasks, but also contributed to all the developers that in the future want to use the middleware to create a connection between \textit{Botkit} and \textit{Dialogflow}.\\\\
	Finally, we want to thank our professor \textit{Raphaël Troncy} and our supervisors \textit{Pasquale Lisena} and \textit{Thibault Ehrhart} for the great way in how they followed our work, provided help and, overall, in how they gave us suggestions and solutions to the various issues we've encountered in this path.
	